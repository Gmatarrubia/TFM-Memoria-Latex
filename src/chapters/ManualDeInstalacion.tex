\chapter{Manual de Instalación.}\label{sec:ManualDeInstalacion}

\paragraph{}En este capítulo se va a explicar cómo instalar el entorno de desarrollo
para cada uno de los principales sistemas operativos. Los pasos descritos se podrán
seguir con independencia del estado previo del sistema. No se asume ningún estado previo.

\section{Consideraciones previas}

\paragraph{}El sistema operativo y sus principales tecnologías y herramientas tienen
como base su uso en sistemas operativos GNU/Linux, concretamente la versión 20.04 de
la distribución de Ubuntu. Por lo que, a pesar de poder ser utilizados en cualquier
sistema, serán en éstos donde será más fácil y óptimo su uso.

\paragraph{} Al igual que sucede en otros capítulos de este documento, se van a diferenciar
 las dos partes lógicas que componen el proyecto software.

\section{Instalación del entorno Flutter}

\paragraph{}En el entorno Flutter vamos a ser capaces de desarrollar la aplicación,
pasar sus test y compilarla para diferentes plataformas.

\subsection{Instalación en sistemas operativos GNU/Linux}

\paragraph{}Para la instalación de este entorno en un sistema operativo basado en
la versión de Ubuntu 20.04 se podrá optar por 2 estrategias: correr el entorno de
forma nativa o utilizar el método basado en docker.

\paragraph{Manera nativa:}
\begin{enumerate}
    \item Instalar la herramienta \gls{git} de control de versiones.
    \begin{lstlisting}[style=consola, numbers=left]
        $ sudo apt install git
    \end{lstlisting}

    \item Clonar el repositorio desde github
    \begin{lstlisting}[style=consola, numbers=left]
        $ git clone https://github.com/Gmatarrubia/rpi_weather.git
    \end{lstlisting}

    \item Ejecutar el script de instalación de paquetes. Este paso solamente es necesario
    ejecutarlo una vez.
    \begin{lstlisting}[style=consola, numbers=left]
        $ ./installDevEnv.sh
    \end{lstlisting}

    \item Ejecutar de opbtención de dependencias. Este script solamente será necesario
    ejecutarlo una vez, a no se que se hayan borrado las dependencias.
    \begin{lstlisting}[style=consola, numbers=left]
        $ ./getSources.sh
    \end{lstlisting}

\end{enumerate}


\subsection{Instalación en sistemas operativos Windows}

\paragraph{}Blabla

\subsection{Instalación en sistemas operativos MacOS}

\paragraph{}Blabla

\section{Instalación del entorno Yocto}

\subsection{Instalación en sistemas operativos GNU/Linux}

\paragraph{}Blabla

\subsection{Instalación en sistemas operativos Windows}

\paragraph{}Blabla

\subsection{Instalación en sistemas operativos MacOS}

\paragraph{}Blabla

%las referencias a artculos se ponen con \cite,
%las referencias a imgenes \ref,
%las referencias a glosario \gls,
%y las referencias a ecuaciones \eqref