\chapter{Conclusiones}\label{sec:conclusiones}

\paragraph{}Tras haber presentado los resultados del proyecto, vamos a terminar extrayendo
los aprendizajes y conclusiones más importantes. Para ello se va a ir repasando cada
objetivo marcado y se va a ir haciendo una valoración sobre el grado de consecución.
Se continuará hablando sobre los aprendizajes que este proyecto ha supuesto. Y por
último se van a presentar las opciones principales que se plantean como consecución al
trabajo realizado.

\section{Cumplimiento de Objetivos principales}

\paragraph{\checkmark Diseño e implementación del entorno de desarrollo:} Objetivo
cumplido, hemos construido un entorno de desarrollo totalmente funcional. Incluso se
podría considerar que finalmente han sido dos los entornos creados, uno para el entorno
de Flutter y otro para el entorno de Yocto.

\paragraph{\checkmark Diseño e implementación del ciclo de vida del desarrollo software.}
Como se ha descrito en los anteriores apartados se han diseñado las estrategias de desarrollo
y el hipotético uso de la infrestuctura necesario.

\paragraph{\checkmark Explicación del uso del entorno de desarrollo según los roles del
equipo de desarrolladores:} Objetivo cumplido, ya que se ha explicado como cada uno de
los roles interactuarán con el entorno y cuales serán las herramientas más utilizadas
en cada caso.

\paragraph{\checkmark Diseño e implementación del software de sistema:} Se ha conseguido
un software de sistema que ha resultado ser una distribución basada en linux generada
con Yocto, totalmente funcional, que permita la ejecución y uso de la aplicación embebida.

\paragraph{\checkmark Diseño e implementación de una aplicación de meteorología (frontend y
backend) de características básicas:} La aplicación que hemos llamado \emph{Rpi Weather}
es una aplicación monolítica (frontend y backend) que cumple con los casos de uso
requisitados. La elección de esta aplicación tenía intención de ser un demostrador y
como tal considero que cumple perfectamente.

\paragraph{\checkmark Procesos de testing y deployment de software manuales:}Hemo definido
sendas acciones para cada entorno y hemos conseguido hacerlas de manera manual. De esta
forma el desarrollador toma la responsabilidad de probar sus cambios antes de intentar
integrarlos.

\section{Cumplimiento de Objetivos secundarios}

\paragraph{\checkmark Análisis y justificación de las decisiones tomadas:}Considero que
a lo largo de la memoria se han ido explicando las decisiones que se han ido tomando y
que el lector ha podido comprender el por qué de las decisiones tomadas. De hecho, espero
haber podido convencer al lector de probar por su cuenta el entorno.

\paragraph{\checkmark Uso de herramientas de código abierto:} Objetivo cumplido, a continuación
desgloso un listado de todas las herramientas utilizadas así como las licencias que
utilizan.

\begin{table}[H]
    \begin{center}
    \begin{tabular}{|c|c|}
    \hline
    Yocto & GNU General Public License version 2.0 \\
    \hline
    Flutter & New BSD License \\
    \hline
    Ansible & GNU General Public License \\
    \hline
    VSCode & standard MIT license \\
    \hline
    Docker &  Docker Subscription Service Agreement* \\
    \hline
    \end{tabular}
    \end{center}
    \caption{Ejemplo de tabla}\label{tab:table_example}
\end{table}

\paragraph{}
\emph{Aclaración licencia de docker:} Docker se distribuye bajo una licencia privativa,
no obstante, su uso es gratuito para: uso personal o proyectos de código abierto. En
caso de empresas de más de 250 personas y/o de más de 10 millones de beneficio anual,
su uso no es gratuito.

\paragraph{$\times$ Ampliación en las características demostradas con la aplicación de
meteorología:} No considero que se haya cumplido este objetivo, ya que la aplicación
no ha incluido ninguna característica adicional a las previamente marcadas.

\paragraph{$\times$ Procesos de testing y deployment de software automátizados:} Considero
que este objetivo secundario no se ha conseguido totalmente ya que aunque se ha hecho
una diseño y una propuesta de infrestuctura de testing no ha llegado a implantarse.

\section{Aprendizajes}

\paragraph{}Blabla

\section{Continuación del desarrollo}

\paragraph{}Blabla

%las referencias a artículos se ponen con \cite,
%las referencias a glosario \gls,
%y las referencias a ecuaciones \eqref
%las referencias a imgenes, tablas o figuras o secciones
% se ponen con \ref (sólo número) o con \hyperref[sec:X]{Blabla}
