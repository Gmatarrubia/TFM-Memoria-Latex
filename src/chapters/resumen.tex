%chapter introduce un nuevo capítulo
\chapter{Resumen}

\paragraph{}Este trabajo pretende recoger los aspectos de diseño e implementación
tenidos en cuenta para la creación de un entorno de desarrollo, bajo unos requisitos
tecnológicos y empresariales específicos. Explicará cómo debe ser usado por los
desarrolladores, teniendo en cuenta el ciclo de vida de desarrollo de software (SDLC,
por sus siglas en inglés). Además, se presenta el desarrollo software de una aplicación
de meteorología en un dispositivo embebido Raspberry Pi, que servirá de ejemplo de
los procesos descritos.

\paragraph{}Los requisitos tecnológicos marcados, representan una situación ficticia
que podría darse en una nueva empresa de rápido crecimiento o \textit{start up} dedicada
al ámbito del IoT o sistemas embebidos. Los requisitos son los siguientes:
pequeña empresa en rápido crecimiento, equipo multidisciplinar, se quiere un entorno
flexible y dínamico accesible para adaptarse a rápidamente a cambios de hardware y
a las decisiones empresariales y estratégicas de operaciones. Además se quiere basar
el proyecto en tecnologías que permitan esta rápida adaptación, que sean herramientas
de uso universal, en la medida de lo posible, y de código abierto.

\paragraph{Palabras clave:} Entornos de desarrollo, Yocto, Flutter, sistemas embebidos,
 Raspberry Pi, Qemu, IoT, SDLC.

\chapter{Abstract}

\paragraph{}This document aims to talk about the design and implementation aspects in
the creation of an development environment with some technological and business requirements.
It explains how it should be used by developers in their journy, taking in consideration
the software development life cyle(SDLC). Also, It shows the development of an
metherological application for an embedded device. In this case, a Raspberry Pi. It
is used as example for the process showed previously.

\paragraph{} The technological requirements are the following: A start up in the field
of IoT or embeded devices wants and development environment. It has to be for a fast-growing
and multidisciplinary team. It must be flexible enough for adapting to the business
and strategical choices. It should be universal and open source code based, as much as
it could be possible.

\paragraph{Key words:} IDE, Development environment, Yocto Project, Flutter, embedded
systems, Raspberry Pi, Qemu, IoT, SDLC.