\chapter{Guía de Uso.}\label{sec:GuiaDeUso}

\paragraph{}En este capítulo se van a explicar las principales funciones del entorno de
desarrollo, así como las principales opciones de cada herramienta provista a los
desarrolladores.

\section{Consideraciones previas}

\paragraph{}Todos los pasos de todas las guías de este capítulo podrán ser seguidas si
previamente se han seguido las indicaciones del cápitulo \ref{sec:ManualDeInstalacion}.

\paragraph{}El entorno de desarrollo está separado del código de la aplicación
que se usa de ejemplo. Esta separación lógica, tiene sentillo ya que ambas partes pueden
funcionar tanto por separado como juntas. Y ambas partes pueden ser utilizadas para
propositos distintos que el de funcionar juntas.

\section{Entorno de desarrollo Yocto}

\paragraph{}Todo el código necesario para seguir esta guía puede encontrarse en el
siguiente repositorio:
\href{https://github.com/Gmatarrubia/dev_env_rpi_flutter_yocto}{github.com/Gmatarrubia/dev env rpi flutter yocto}

\section{Entorno de desarrollo Flutter}

\paragraph{}Todo el código necesario para seguir esta guía puede encontrarse en el
siguiente repositorio:


\href{https://github.com/Gmatarrubia/rpi_weather}{github.com/Gmatarrubia/rpi weather}

%las referencias a artculos se ponen con \cite,
%las referencias a imgenes \ref,
%las referencias a glosario \gls,
%y las referencias a ecuaciones \eqref
