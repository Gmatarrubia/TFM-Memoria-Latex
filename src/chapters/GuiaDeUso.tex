\chapter{Guía de Uso.}\label{sec:GuiaDeUso}

\paragraph{}En este capítulo se van a explicar las principales funciones del entorno de
desarrollo, así como las principales opciones de cada herramienta provista a los
desarrolladores.

\section{Consideraciones previas}

\paragraph{}Todos los pasos de todas las guías de este capítulo podrán ser seguidas si
previamente se han seguido las indicaciones del cápitulo \ref{sec:ManualDeInstalacion}.

\paragraph{}Al igual que en el capítulo anteriór \ref{sec:ManualDeInstalacion}, la guía
de uso estará dividida en: entorno de desarrollo Flutter y entorno de desarrollo Yocto.
Esta separación lógica, tiene sentillo ya que ambas partes pueden funcionar tanto por
separado como juntas. Y ambas partes pueden ser utilizadas para propositos distintos
que el de funcionar juntas.

\section{Entorno de desarrollo Flutter}

\paragraph{}Blabla


\section{Entorno de desarrollo Yocto}

\paragraph{}Este es el entorno de desarrollo de la distribución FlutterPI. Una distribución
con el kernel de linux para Raspberry Pi basada en poky (yocto) la cuál está pensada
para correr aplicaciones de Flutter. Esto es muy útil para múltiples aplicaciones como
por ejemplo: Máquinas de vending, paneles de información, y muchísimos otros tipos de
HMIs. Este entorno de desarrollo te permite tomarlo de plantilla y desarrollar rápidamente
un sistema HMI, fiable, replicable y altamente personalizable.

\subsection{Características principales}

\paragraph{}Este entorno ha sido diseñado teniendo en cuanta los siguiente aspectos:

\begin{itemize}
    \item Compatibilidad con la mayoría de sistemas operativos.
    \item Facilidad de uso.
    \item Entendible y bien documentado.
    \item Fácil de gestionar remotamente.
    \item Flexibilidad y multipropósito.
\end{itemize}

\subsection{Usos principales}

\paragraph{Nota:} Este entorno corre nativamente en sistemas con distribuciones Ubuntu 20.04.
Para el resto de sistemas utilizar cualquiera de los métodos alternativos:

\begin{itemize}
    \item Entorno dockerizado.
    \item Máquina virtual con Ubuntu 20.04.
    \item Entorno en WSL2 (sólo para sistemas Windows 10/11.)
\end{itemize}

\paragraph{}Acudir al capítulo \hyperref[sec:ManualDeInstalacion]{Manual de instalación}
\ref{sec:ManualDeInstalacion} antes de intentar los procedimientos descritos en este
capítulo.

\subsection{Build}

\paragraph{}Para lanzar la generación de la imagen completa con las configuraciones por
defecto utilizar el siguiente comando. El resultado será una imagen arrancable en una
Raspberry Pi el cuál arrancará una aplicación Flutter al comienzo.

\begin{lstlisting}[style=consola, numbers=left]
    $ ./build.sh
\end{lstlisting}

\paragraph{}Antes de empezar la generación, el entorno te preguntará la configuración
Wi-Fi para conectarse a un punto de acceso (ssid + pass). Ésto se puede saltar si
previamente se han configurado las variables de entorno: \emph{WIFISSID} y \emph{WIFIPASS}.

\paragraph{}Para más información sobre los aspectos del script de build pueden ser
consultados con el comando:

\begin{lstlisting}[style=consola, numbers=left]
    $ ./build.sh --help
\end{lstlisting}

\subsection{Comando de build personalizados}

Build script let you enter your custom bitbake commands. This can be done by using the -bc or --bitbake-cmd argument followed by the double-quoted command. See some examples:

\begin{lstlisting}[style=consola, numbers=left]
    $ ./build.sh --bitbake-cmd "bitbake -s | grep flutter"
    $ ./build.sh -bc "bitbake -D wifi -c clean"
    $ ./build.sh --bitbake-cmd "bitbake-layers show-layers"
\end{lstlisting}

\subsection{Lanzar el modo Sesión Interactiva}

This is a powerful way to debug and develop either your recipes or your flutter apps. If you are interested in open an interative session run this:

\begin{lstlisting}[style=consola, numbers=left]
    $ ./build.sh --shell
\end{lstlisting}

Once you are inside the `shell` you will be able to use commands like:

\begin{itemize}
    \item bitbake
    \item bitbake-getvar
    \item bitbake-layers
    \item devtool
\end{itemize}

\subsection{Cleaning}

\begin{lstlisting}[style=consola, numbers=left]
    $ ./cleanAll.sh
    # Keep in mind it could be dangerous in case you have unsaved changes.
    $ git clean -fdx
\end{lstlisting}


%las referencias a artículos se ponen con \cite,
%las referencias a glosario \gls,
%y las referencias a ecuaciones \eqref
%las referencias a imgenes, tablas o figuras o secciones
% se ponen con \ref (sólo número) o con \hyperref[sec:X]{Blabla}
