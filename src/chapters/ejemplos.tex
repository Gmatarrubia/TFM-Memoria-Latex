\chapter{Cmo escribir en Latex}

\section{Citas}

%las referencias a artculos se ponen con \cite,
%las referencias a imgenes \ref,
%las referencias a glosario \gls,
%y las referencias a ecuaciones \eqref

Esto es un ejemplo de cita de un artculo \cite{Brunete:2013}.


\section{Listas}

%itemize es una lista. Cada trmino lleva delante un \item
Ejemplo de lista de puntos:
\begin{itemize}
\item Ejemplo1.
\item Ejemplo2.
\end{itemize}

Y lista numerada:
\begin{enumerate}
\item Elemento 1
\item Elemento 2
\end{enumerate}

\section{Tablas}

Ejemplo de tabla. Como se aprecia en la tabla \ref{tab:table_example}...
\begin{table}[tb]
\caption{Ejemplo de tabla}
\label{tab:table_example}
\begin{center}
\begin{tabular}{|c||c|c|}
\hline
One & Two & Three\\
\hline
F1A & F1B & F1C\\
F2A & F2B & F2C\\
\hline
\end{tabular}
\end{center}
\end{table}

\section{Referencia a una seccin}
\label{sec:refsec}

Ejemplo de referencia a la seccin \ref{sec:refsec}

\section{Texto}

Testo en \textbf{negrita} y \textit{cursiva}.

\section{Figuras}

Ejemplo de referencia a figura (figura \ref{fig:logo_upm}). Es importante que todas las figs que aparezcan estn referenciadas, as como las tablas. En general las figuras se colocarn al principio o al final de cada pgina ([tb] en latex), a no ser que por alguna necesidad se deban colocar en una posicin exacta ([h]).

%caption es el pie de foto, y label es el nombre que se da a la imagen para referenciarla despus. label no puede llevar acentos y no se muestra de cara al documento final (es slo interno).
\begin{figure}[tb]
\centering
\includegraphics[width=0.45\textwidth]{figs/Logo_UPM.jpg}
\caption{Logotipo de la UPM}
\label{fig:logo_upm}
\end{figure}

% Ejemplo código consola
\begin{lstlisting}[style=consola, numbers=left]
    #!/bin/bash

    #Genera las clases para el Cliente_py y el Servidor_rbpi

    protoc -I=. --python_out=../Cliente_py mensaje.proto
    protoc -I=. --cpp_out=../Servidor_rbpi mensaje.proto

    echo "Se han generado los archivos mensaje.pb.h y mensaje.pb.cc"
\end{lstlisting}

% Ejemplo código en C
\begin{lstlisting}[style=C, numbers=none]
    cmake_minimum_required(VERSION 3.6)
    project(Servidor_rbpi)

    set(SOURCE_FILES main.cpp vlc_object.cpp conn_zmq.cpp mensaje.pb.cc gestorMP.cpp latidos_zmq.cpp lst_medias.cpp setIP.h zmqvideo.cpp)
    set(HEADER_FILES conn_zmq.h vlc_object.h mensaje.pb.h gestorMP.h latidos_zmq.h lst_medias.h rpiGPIO.h setIP.h zmqvideo.h)
    add_executable(Servidor_rbpi ${SOURCE_FILES})
    target_link_libraries(Servidor_rbpi ${LIBVLC_LIBRARY} ${ZMQ_LIBRARY} ${PROTOBUF_LIBRARY}
\end{lstlisting}