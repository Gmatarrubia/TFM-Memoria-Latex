%\newacronym{NAME}{NAME}{Description}
\newacronym{SDK}{SDK}{Software Development Kit}
\newacronym{OS}{OS}{Sistema operativo, por sus siglas en inglés}
\newacronym{UI/UX}{UI/UX}{Interfaz y experiencia de usuario por sus siglas en inglés}
\newacronym{QA}{QA}{Asegurador de calidad, por sus siglas en inglé}
\newacronym{WSL}{WSL}{Windows Subsystem for Linux}
\newacronym{WSL2}{WSL2}{Windows Subsystem for Linux version 2}
\newacronym{FPGAs}{FPGAs}{Dispositivos lógicos programables basado en \emph{arrays} de puertas lógicas}
\newacronym{ADT}{ADT}{Kit de desarrollo de aplicaciones, por sus siglas en inglés}
\newacronym{IoT}{IoT}{Internet de la cosas, por sus siglas en inglés}
\newacronym{SBC}{SBC}{Computadoras de una sola placa, normalmente el tamaño de una tarjeta
bancaria. Suelen ser dispositivos de bajo consumo y de propósito específico}
\newacronym{rpi}{rpi}{Raspberry Pi}
\newacronym{IDE}{IDE}{Entorno de desarrollo integrado por sus siglas en inglés. Son programas
que reunen varias herramientas destinadas al desarrollo de software}
\newacronym{Code}{Code}{Visual Studio Code}
\newacronym{vscode}{vscode}{Visual Studio Code}
\newacronym{cross-compile}{cross-compile}{Compilación cruzada}
\newacronym{IaaC}{IaaC}{Infrasturure as a Code}

%\newglossaryentry{sample}{name={sample}, description={an example}}
\newglossaryentry{bootloader}{name={bootloader},description={Programa de gestión de arranque}}
\newglossaryentry{hipervisor}{name={hipervisor},description={Programa que añade una capa de virtualización
entre el sistema operativo y el microprocesador, se utiliza común mente para ejecutar más de una aplicación
o sistema operativo en un sólo microprocesador, de forma tal que ambas parte piensen que tiene el uso
exclusivo del micro}}
\newglossaryentry{KVM}{name={KVM},description={Módulo del kernel de linux que permite a éste actuar como un hipervisor}}
\newglossaryentry{XEN}{name={XEN},description={hipervisor de tipo 1 desarrollado originalmente por
la Universidad de Cambridge y posteriormente mantenido por la Linux Fundation}}
\newglossaryentry{Linux Foundation}{name={Linux Foundation},description={Es un consorcio tecnológico
sin ánimo de lucro establecido para adoptar el crecimiento de Linux}}
\newglossaryentry{startup}{name={startup},description={Empresa de nueva creación con posibilidad
de un gran crecimiento gracias a la comercialización de productos o servicios de tecnologías
de la información y comunicación}}
\newglossaryentry{SDLC}{name={SDLC},description={ciclo de vida de desarrollo de software,
por sus siglas en inglés}}
\newglossaryentry{Raspberri Pi}{name={Raspberry Pi},description={Popular SBC de bajo coste}}
\newglossaryentry{DevOps}{name={DevOps},description={Es la cultura empresarial que busca una colaboración
más estrecha entre los equipos de desarrollo y operaciones}}
\newglossaryentry{Widgets}{name={Widgets},description={Aplicativo o componente de la interfad
que realiza una función específica}}
\newglossaryentry{compilacion cruzada}{name={compilación cruzada},description={Es aquella compilación
que tiene como objetivo la ejecución del código en una computadora con una arquitectura diferente
a la computadora utilizada para la compilación}}
\newglossaryentry{front-end}{name={front-end},description={En una aquitectura software basada
en capas, nos referimos a las capas que conformen la interfaz como fron-end}}
\newglossaryentry{back-end}{name={back-end},description={En una aquitectura software basada
en capas, nos referimos a las capas de servicios o con la lógica del programa como back-end}}
\newglossaryentry{scripts}{name={scripts},description={Pequeño conjunto de instrucciones
que permite automatizar una tarea repetitiva}}
\newglossaryentry{shell}{name={shell},description={Consola de sistema}}
\newglossaryentry{framework}{name={framework},description={Es un gran conjunto de librerías que
proveen de un gran número de recursos de desarrollo.}}
\newglossaryentry{CI/CD}{name={CI/CD},description={Integración y entrega continua. Por
sus siglas en inglés.}}
\newglossaryentry{on premises}{name={on premises},description={Hace referencia a cuando
un servidor CI/CD es propiedad de la empresa, a nivel hardware, y por tanto no está externalizado
en cloud.}}
\newglossaryentry{git}{name={git},description={Herramienta de control de versiones}}
\newglossaryentry{plugins}{name={plugins},description={Módulo software que añade una funcionalidad
específica y adicional a un programa}}
\newglossaryentry{pipeline}{name={pipeline},description={Documento de texto en lenguaje de marcado
que describe las características de la infraestructura de CI/CD, así como las pruebas o
procesos que sufre el código en dicha infraestructura}}
\newglossaryentry{open source}{name={open source}, description={Es una corriente de software
que aboga por la publicación del código fuente para poder ser mantenido por la comunidad de usuarios}}


\glsaddall
